% Options for packages loaded elsewhere
\PassOptionsToPackage{unicode}{hyperref}
\PassOptionsToPackage{hyphens}{url}
%
\documentclass[
  12pt,
  twoside]{article}
\usepackage{amsmath,amssymb}
\usepackage{lmodern}
\usepackage{setspace}
\usepackage{ifxetex,ifluatex}
\ifnum 0\ifxetex 1\fi\ifluatex 1\fi=0 % if pdftex
  \usepackage[T1]{fontenc}
  \usepackage[utf8]{inputenc}
  \usepackage{textcomp} % provide euro and other symbols
\else % if luatex or xetex
  \usepackage{unicode-math}
  \defaultfontfeatures{Scale=MatchLowercase}
  \defaultfontfeatures[\rmfamily]{Ligatures=TeX,Scale=1}
\fi
% Use upquote if available, for straight quotes in verbatim environments
\IfFileExists{upquote.sty}{\usepackage{upquote}}{}
\IfFileExists{microtype.sty}{% use microtype if available
  \usepackage[]{microtype}
  \UseMicrotypeSet[protrusion]{basicmath} % disable protrusion for tt fonts
}{}
\makeatletter
\@ifundefined{KOMAClassName}{% if non-KOMA class
  \IfFileExists{parskip.sty}{%
    \usepackage{parskip}
  }{% else
    \setlength{\parindent}{0pt}
    \setlength{\parskip}{6pt plus 2pt minus 1pt}}
}{% if KOMA class
  \KOMAoptions{parskip=half}}
\makeatother
\usepackage{xcolor}
\IfFileExists{xurl.sty}{\usepackage{xurl}}{} % add URL line breaks if available
\IfFileExists{bookmark.sty}{\usepackage{bookmark}}{\usepackage{hyperref}}
\hypersetup{
  pdftitle={Can R Notebooks help with reproducibility?},
  pdfauthor={Kevin Ha; Ola Andre Olofsson},
  hidelinks,
  pdfcreator={LaTeX via pandoc}}
\urlstyle{same} % disable monospaced font for URLs
\usepackage[margin=1in]{geometry}
\usepackage{graphicx}
\makeatletter
\def\maxwidth{\ifdim\Gin@nat@width>\linewidth\linewidth\else\Gin@nat@width\fi}
\def\maxheight{\ifdim\Gin@nat@height>\textheight\textheight\else\Gin@nat@height\fi}
\makeatother
% Scale images if necessary, so that they will not overflow the page
% margins by default, and it is still possible to overwrite the defaults
% using explicit options in \includegraphics[width, height, ...]{}
\setkeys{Gin}{width=\maxwidth,height=\maxheight,keepaspectratio}
% Set default figure placement to htbp
\makeatletter
\def\fps@figure{htbp}
\makeatother
\setlength{\emergencystretch}{3em} % prevent overfull lines
\providecommand{\tightlist}{%
  \setlength{\itemsep}{0pt}\setlength{\parskip}{0pt}}
\setcounter{secnumdepth}{-\maxdimen} % remove section numbering
\ifluatex
  \usepackage{selnolig}  % disable illegal ligatures
\fi
\newlength{\cslhangindent}
\setlength{\cslhangindent}{1.5em}
\newlength{\csllabelwidth}
\setlength{\csllabelwidth}{3em}
\newenvironment{CSLReferences}[2] % #1 hanging-ident, #2 entry spacing
 {% don't indent paragraphs
  \setlength{\parindent}{0pt}
  % turn on hanging indent if param 1 is 1
  \ifodd #1 \everypar{\setlength{\hangindent}{\cslhangindent}}\ignorespaces\fi
  % set entry spacing
  \ifnum #2 > 0
  \setlength{\parskip}{#2\baselineskip}
  \fi
 }%
 {}
\usepackage{calc}
\newcommand{\CSLBlock}[1]{#1\hfill\break}
\newcommand{\CSLLeftMargin}[1]{\parbox[t]{\csllabelwidth}{#1}}
\newcommand{\CSLRightInline}[1]{\parbox[t]{\linewidth - \csllabelwidth}{#1}\break}
\newcommand{\CSLIndent}[1]{\hspace{\cslhangindent}#1}

\title{Can R Notebooks help with reproducibility?}
\author{Kevin Ha \and Ola Andre Olofsson}
\date{}

\begin{document}
\maketitle

\setstretch{1.5}
\hypertarget{introduction}{%
\section{Introduction}\label{introduction}}

In this document we will look at reproducibility and the importance of
it. In this context, we will address the topic of using ``R Notebook''
in RStudio. We'll look into theory regarding reproducibility and R
Notebook, which we will also follow up with a discussion.

\hypertarget{what-is-reproducibility}{%
\subsection{What is reproducibility?}\label{what-is-reproducibility}}

In science it's by some considered to be the holy grail of statistics.
Basically it means doing an experiment or research multiple times by
different people or groups, with the same numbers and get the same
result. This is desirable because it is important that the findings are
robust, trustworthy, and conducted in a satisfactory way.

Researchers often use the terms replicability and reproducibility
interchangeably, but it is useful to distinguish between them.
Replicability is ``re-performing the experiment and collecting new
data,'' whereas reproducibility is ``re-performing the same analysis
with the same code using a different analyst'' Patil, Peng, and Leek
(2016). According to Stanovich (2014), one of the most important
criteria for scientists is that the findings are presented in a way that
they can be replicated, exposed for criticism, or extended further on.

\hypertarget{challenges}{%
\subsection{Challenges}\label{challenges}}

One of the big challenges of Data science has been to reproduce the
identical results from an earlier research. In a significant amount of
research it has proved hard or flat out impossible to reproduce the
results from other peoples and older research for a multitude of
reasons. One of the big reason is quite simply that the data used in the
research is unobtainable, due to them being ``too old'' or lost. Another
reason is the author doesn't want to share the data or can't find it
(again, lost). Other challenges is the usage of different systems,
equipment, and so on. Theories around the research and calculations can
also change over time.

\hypertarget{computable-documents}{%
\subsection{Computable Documents}\label{computable-documents}}

In more modern times a new idea around this have been brought up:
computable documents. The idea behind this is when publishing a research
paper, one submits the data, equations and calculations together with a
document that's computable. That way when other researches wants to run
the experiment, they have access to everything, and can see what others
did before them.

Computable Document Format (CDF) is an electronic document format
designed to allow authoring dynamically generated, interactive content
{``Computable {Document Format}''} (2021). The format was designed to
emprove or kill the PDF by making a document the reader could interact
with, by using sliders, menus, and buttons which PDF does not allow.
This consequently increases engagement in the readers and their
understanding.

So the question then becomes: can computable documents like R Notebook
help with reproducibility?

\hypertarget{short-literature-review}{%
\section{Short literature review}\label{short-literature-review}}

According to McNutt (2014) advancing in science is reliant on
discoveries that can be trusted, but stated that studies that have been
performed, can't be reproduced. In tandum with this, Peng Peng (2011)
has brought up the idea that reproducibility should be a requirement for
publishing research.

Already in the early 1930's the accessability to the data for a research
was aired by Frisch He stated that in statistical research, the raw data
should be published Frisch (1933). Later in the 1960's it was deemed
nearly impossible to reproduce research on economy with big models. Then
in 1982 there was a research project done: the Journal of Money, Credit
and Banking, where they tried to replicate research article with only
the submitted data. This is looking back, very close to reproducibility
in modern terms. The results was a staggering 2 of 70 articles could be
reproduced. They concluded was because for the most part missing data,
documentation and computer systems, as quite few of the authors would
provide their data and coding Dewald, Thursby, and Anderson (1986). To
counteract this, they came up the solution to store the coding and raw
data used in research articles, which in a sense is the foundation of
computable documents today.

\hypertarget{discussion}{%
\section{Discussion}\label{discussion}}

Reproducibility is one important approach that scientists use to gain
confidence in their conclusions McNutt (2014). How confident in a
conclusion would one be if the scientist could not reproduce the result
and conclude similarly again? She states that this confidence is
important regarding the broad scientific community. This is due to the
scientifical knowledge is public in a special sense. It does not only
exist in the mind of the particular researcher, but one could argue that
the knowledge does not exist until it has been submitted to the
scientific community for criticism and empirical testing Stanovich
(2014). He elucidates this matter by introducing the term replication.
The term is understood to mean that a finding must be presented to the
scientific community in a way that allows fellow scientists to re-do the
study and achieve the same results.

According to a analysis conducted by McCullough (2009), most economics
journals take no substantive measures to ensure that the results they
publish are replicable. Top economics journals have been adopting
mandatory data+code archives in the past few years. The movement toward
mandatory data+code archives has yet to reach the open access journals.
Open Access (OA) journals are often perceived, rightly or wrongly, as
having a second-class status compared to traditional journals. Note that
in the list of top 50 journals in Table 1, not a single journal is OA.

Some important points to highlight that can cause problems with
reproducibility in R Notebook is:

\begin{enumerate}
\def\labelenumi{\arabic{enumi}.}
\tightlist
\item
  System or version incompatibility
\end{enumerate}

\begin{itemize}
\tightlist
\item
  Packages, software, etc.
\end{itemize}

\begin{enumerate}
\def\labelenumi{\arabic{enumi}.}
\setcounter{enumi}{1}
\item
  Advancement or changes in theories
\item
  Human error
\end{enumerate}

\begin{itemize}
\tightlist
\item
  May only be reduced to a certain degree
\item
  Forgetting to add information and data
\end{itemize}

\begin{enumerate}
\def\labelenumi{\arabic{enumi}.}
\setcounter{enumi}{3}
\tightlist
\item
  Motivation
\end{enumerate}

\begin{itemize}
\tightlist
\item
  Certainty relies on willingness
\item
  Willingness to turn in all data, code, information, etc.
\end{itemize}

Firstly, there are different OS and OS platforms, which can create
problems. Software and packages being updated can cause incompatibility
and across different versions. To fix this one needs a huge systematic
archive of downloadable software and packages as it was at the time.
Another issue is that a theory about a subject today can change in 20 --
30 years, which can affect how we go about it and in worst case,
calculations and formulas can change.

There's also the human factor. Human error such as a misunderstanding of
the subject or a typo and so on can create follow up mistakes through
the document. In huge documents with 1000s of observations and multiple
formulas and calculations can make a small mistake into a big one that
can be hard to track and fix, which makes reproducibility borderline
impossible.

Finally, there's also willingness. Creating a big R Notebook or
computable document with 1000s of observations, formulas, calculations,
analysis, creating data sets and sourcing it all can be a huge workload.
If the person or group takes a shortcut or leaves something out because
they think it's not necessary to write down or forgets to add it, we're
once again missing information. Without that information we can't
reproduce the research, and if we must ask the author for the data and
information they didn't add, part of the point with computable documents
for reproducing is lost. On top of that, we're relying on the author(s)
actually having the data on hand or remembering it correctly after
potentially years.

\textbf{Why is reproducibilty important? What are the issues surrounding
it? What steps have been taken to improve the situation? What does
scientist say about it?}

\hypertarget{conclusion}{%
\section{Conclusion}\label{conclusion}}

The question was «can R Notebook help with reproducibility?». The short
answer is yes, it can help, but it doesn't solve the problem, atleast
not yet. There's quite a few issues with the current R Notebook system
and computable documents.

To summarize, computable documents like R Notebook definitly helps with
reproducibilty, but it doesn't solve the issue, atleast not yet. There
are currently too many problems both human and technical. There's
progress in these areas like using identical computer setups from a
server, libraries of software and packages, as they were, are currently
being developed, the option to check the system for the setup it used
when the research was first done, and so on. So while it doesn't fix the
issue, it jelps and it's definitly a step in the right direction.

\hypertarget{reference}{%
\section{Reference}\label{reference}}

\hypertarget{refs}{}
\begin{CSLReferences}{1}{0}
\leavevmode\hypertarget{ref-ComputableDocumentFormat2021}{}%
{``Computable {Document Format}.''} 2021. \emph{Wikipedia}, July.

\leavevmode\hypertarget{ref-dewaldReplicationEmpiricalEconomics1986}{}%
Dewald, William G., Jerry G. Thursby, and Richard G. Anderson. 1986.
{``Replication in {Empirical Economics}: {The Journal} of {Money},
{Credit} and {Banking Project}.''} \emph{The American Economic Review}
76 (4): 587--603.

\leavevmode\hypertarget{ref-frischEditorNote1933}{}%
Frisch, Ragnar. 1933. {``Editor's {Note}.''} \emph{Econometrica} 1 (1):
1--4.

\leavevmode\hypertarget{ref-mcculloughOpenAccessEconomics2009}{}%
McCullough, B. D. 2009. {``Open {Access Economics Journals} and the
{Market} for {Reproducible Economic Research}.''} \emph{Economic
Analysis and Policy} 39 (1): 117--26.
\url{https://doi.org/10.1016/S0313-5926(09)50047-1}.

\leavevmode\hypertarget{ref-mcnuttReproducibility2014}{}%
McNutt, Marcia. 2014. {``Reproducibility.''} \emph{Science} 343 (6168):
229--29. \url{https://doi.org/10.1126/science.1250475}.

\leavevmode\hypertarget{ref-patilStatisticalDefinitionReproducibility2016}{}%
Patil, Prasad, Roger D. Peng, and Jeffrey T. Leek. 2016. {``A
Statistical Definition for Reproducibility and Replicability.''} {Cold
Spring Harbor Laboratory}. \url{https://doi.org/10.1101/066803}.

\leavevmode\hypertarget{ref-pengReproducibleResearchComputational2011}{}%
Peng, Roger D. 2011. {``Reproducible {Research} in {Computational
Science}.''} \emph{Science} 334 (6060): 1226--27.
\url{https://doi.org/10.1126/science.1213847}.

\leavevmode\hypertarget{ref-stanovichHowThinkStraight2014}{}%
Stanovich, Keith E. 2014. \emph{How to Think Straight about Psychology}.
Ninth edition, Pearson new international edition. {Harlow, Essex}:
{Pearson{}}.

\end{CSLReferences}

\hypertarget{appendix}{%
\section{Appendix}\label{appendix}}

\includegraphics{data/Test_bilde.PNG}

\end{document}
